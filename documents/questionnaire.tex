\documentclass[a4paper,10pt,oneside,headsepline]{scrartcl}
\usepackage[ngerman]{babel}
\usepackage[utf8]{inputenc}
\usepackage{wasysym}% provides \ocircle and \Box
\usepackage{enumitem}% easy control of topsep and leftmargin for lists
\usepackage{color}% used for background color
\usepackage{forloop}% used for \Qrating and \Qlines
\usepackage{ifthen}% used for \Qitem and \QItem
\usepackage{typearea}
\usepackage{tikz}

 \areaset{17cm}{26cm}
 \setlength{\topmargin}{-1cm}
\usepackage{scrlayer-scrpage}
\pagestyle{scrheadings}
\ohead{\pagemark} 
\chead{}
\cfoot{}

%%%%%%%%%%%%%%%%%%%%%%%%%%%%%%%%%%%%%%%%%%%%%%%%%%%%%%%%%%%%
%% Beginning of questionnaire command definitions %%
%%%%%%%%%%%%%%%%%%%%%%%%%%%%%%%%%%%%%%%%%%%%%%%%%%%%%%%%%%%%
%%
%% 2010, 2012 by Sven Hartenstein
%% mail@svenhartenstein.de
%% http://www.svenhartenstein.de
%%
%% Please be warned that this is NOT a full-featured framework for
%% creating (all sorts of) questionnaires. Rather, it is a small
%% collection of LaTeX commands that I found useful when creating a
%% questionnaire. Feel free to copy and adjust any parts you like.
%% Most probably, you will want to change the commands, so that they
%% fit your taste.
%%
%% Also note that I am not a LaTeX expert! Things can very likely be
%% done much more elegant than I was able to. If you have suggestions
%% about what can be improved please send me an email. I intend to
%% add good tipps to my website and to name contributers of course.
%%
%% 10/2012: Thanks to karathan for the suggestion to put \noindent
%% before \rule!

%% \Qq = Questionaire question. Oh, this is just too simple. It helps
%% making it easy to globally change the appearance of questions.
\newcommand{\Qq}[1]{\textbf{#1}}

%% \QO = Circle or box to be ticked. Used both by direct call and by
%% \Qrating and \Qlist.
\newcommand{\QO}{$\Box$}% or: $\ocircle$

%% \Qrating = Automatically create a rating scale with NUM steps, like
%% this: 0--0--0--0--0.
\newcounter{qr}
\newcommand{\Qrating}[1]{\QO\forloop{qr}{1}{\value{qr} < #1}{---\QO}}

%% \Qline = Again, this is very simple. It helps setting the line
%% thickness globally. Used both by direct call and by \Qlines.
\newcommand{\Qline}[1]{\noindent\rule{#1}{0.6pt}}

%% \Qlines = Insert NUM lines with width=\linewith. You can change the
%% \vskip value to adjust the spacing.
\newcounter{ql}
\newcommand{\Qlines}[1]{\forloop{ql}{0}{\value{ql}<#1}{\vskip0em\Qline{\linewidth}}}

%% \Qlist = This is an environment very similar to itemize but with
%% \QO in front of each list item. Useful for classical multiple
%% choice. Change leftmargin and topsep accourding to your taste.
\newenvironment{Qlist}{%
\renewcommand{\labelitemi}{\QO}
\begin{itemize}[leftmargin=1.5em,topsep=-.5em]
}{%
\end{itemize}
}

%% \Qtab = A "tabulator simulation". The first argument is the
%% distance from the left margin. The second argument is content which
%% is indented within the current row.
\newlength{\qt}
\newcommand{\Qtab}[2]{
\setlength{\qt}{\linewidth}
\addtolength{\qt}{-#1}
\hfill\parbox[t]{\qt}{\raggedright #2}
}

%% \Qitem = Item with automatic numbering. The first optional argument
%% can be used to create sub-items like 2a, 2b, 2c, ... The item
%% number is increased if the first argument is omitted or equals 'a'.
%% You will have to adjust this if you prefer a different numbering
%% scheme. Adjust topsep and leftmargin as needed.
\newcounter{itemnummer}
\newcommand{\Qitem}[2][]{% #1 optional, #2 notwendig
\ifthenelse{\equal{#1}{}}{\stepcounter{itemnummer}}{}
\ifthenelse{\equal{#1}{a}}{\stepcounter{itemnummer}}{}
\begin{enumerate}[topsep=2pt,leftmargin=2.8em]
\item[\textbf{\arabic{itemnummer}#1.}] #2
\end{enumerate}
}

%% \QItem = Like \Qitem but with alternating background color. This
%% might be error prone as I hard-coded some lengths (-5.25pt and
%% -3pt)! I do not yet understand why I need them.
\definecolor{bgodd}{rgb}{0.8,0.8,0.8}
\definecolor{bgeven}{rgb}{0.9,0.9,0.9}
\newcounter{itemoddeven}
\newlength{\gb}
\newcommand{\QItem}[2][]{% #1 optional, #2 notwendig
\setlength{\gb}{\linewidth}
\addtolength{\gb}{-5.25pt}
\ifthenelse{\equal{\value{itemoddeven}}{0}}{%
\noindent\colorbox{bgeven}{\hskip-3pt\begin{minipage}{\gb}\Qitem[#1]{#2}\end{minipage}}%
\stepcounter{itemoddeven}%
}{%
\noindent\colorbox{bgodd}{\hskip-3pt\begin{minipage}{\gb}\Qitem[#1]{#2}\end{minipage}}%
\setcounter{itemoddeven}{0}%
}
}

%%%%%%%%%%%%%%%%%%%%%%%%%%%%%%%%%%%%%%%%%%%%%%%%%%%%%%%%%%%%
%% End of questionnaire command definitions %%
%%%%%%%%%%%%%%%%%%%%%%%%%%%%%%%%%%%%%%%%%%%%%%%%%%%%%%%%%%%%

\begin{document}

\begin{center}
\textbf{\huge Time perception questionnaire }
\end{center}\vskip1em

%\noindent Welcome to this very important survey with which we
%researchers want to look deep inside of you (be afraid!). Anyway,
%thank you for filling it all out.

\section*{About you}

\Qitem{ \Qq{First name}: \Qline{8cm} }

\Qitem{ \Qq{Last name }: \Qline{8cm} }

%\Qitem{ \Qq{How old are you?} I am \Qline{1.5cm} years old.}

\Qitem{ \Qq{What is your occupation?}: \Qline{5.4cm} }

\Qitem{ \Qq{What is your e-mail?}: \Qline{6.2cm} }


%\Qitem{ \Qq{Are you in a good mood right now?} \hskip0.4cm \QO{}
%absolutely \hskip0.5cm \QO{} not really because: \Qline{3cm} }

\clearpage

\newcommand*{\xMin}{0}%
\newcommand*{\xMax}{24}%
\newcommand*{\yMin}{0}%
\newcommand*{\yMax}{6}%

\section*{Perceptions of time}

\Qitem{ \Qq{Draw a curve of how you experienced time yesterday \\ up is fast, down is slow} }

\begin{tikzpicture}[thick,scale=0.5, every node/.style={scale=0.5}]
    \foreach \i in {\xMin,...,\xMax} {
        \draw [very thin,gray] (\i,\yMin) -- (\i,\yMax) node [below] at (\i,\yMin) {$\i$};
    }
    \foreach \i in {\yMin,...,\yMax} {
%        \draw [very thin,gray] (\xMin,\i) -- (\xMax,\i) node [left] at (\xMin,\i) {$\i$};
        \draw [very thin,gray] (\xMin,\i) -- (\xMax,\i) node [right] (\xMin,\i) {};

    }

% \draw [step=1.0,blue, very thick] (0.5,0.5) grid (5.5,4.5);
% \draw [very thick, brown, step=1.0cm,xshift=-0.5cm, yshift=-0.5cm] (0.5,0.5) grid +(5.5,4.5);

\draw [thick, black] (0,3) -- (\xMax, 3);

\end{tikzpicture}


% ---------------------------------------------------------
% Draw one months
% --------------------------------------------------------- 

\newcommand*{\xMinONEMONTH}{0}%
\newcommand*{\xMaxONEMONTH}{30}%
\newcommand*{\yMinONEMONTH}{0}%
\newcommand*{\yMaxONEMONTH}{6}%


\Qitem{ \Qq{Draw a curve of how you experienced this month \\ up is fast, down is slow} }

\begin{tikzpicture}[thick,scale=0.5, every node/.style={scale=0.5}]

    \foreach \i in {\xMinONEMONTH,...,\xMaxONEMONTH} {
        \draw [very thin,gray] (\i,\yMinONEMONTH) -- (\i,\yMaxONEMONTH) node [below] at (\i,\yMin) { \pgfmathparse{int(\i+1)}\pgfmathresult}; 
    }
    
     
    
    \foreach \i in {\yMinONEMONTH,...,\yMaxONEMONTH} {
        \draw [very thin,gray] (\xMinONEMONTH,\i) -- (\xMaxONEMONTH,\i) node [right] (\xMinONEMONTH,\i) {};
    }

\draw [thick, black] (0,3) -- (\xMaxONEMONTH, 3);

\end{tikzpicture}




% ---------------------------------------------------------
% Draw the months
% --------------------------------------------------------- 

\newcommand*{\xMinMONTH}{0}%
\newcommand*{\xMaxMONTH}{12}%
\newcommand*{\yMinMONTH}{0}%
\newcommand*{\yMaxMONTH}{6}%


\Qitem{ \Qq{Draw a curve of how you experienced last year \\ up is fast, down is slow} }

\begin{tikzpicture}[thick,scale=0.5, every node/.style={scale=0.5}]

    \foreach \i in {\xMinMONTH,...,\xMaxMONTH} {
        \draw [very thin,gray] (\i,\yMinMONTH) -- (\i,\yMaxMONTH) ;
    }
    
     \foreach \x [count=\i] in {JAN, FEB, MAR, APR, MAY, JUN, JUL, AUG, SEP, OCT, NOV, DEC} {
        \draw [very thin,gray] (\i-1, \yMinMONTH) -- (\i-1, \yMaxMONTH) node [below] at (\i-1, \yMinMONTH) {\x};
    }
    
    \foreach \i in {\yMinMONTH,...,\yMaxMONTH} {
        \draw [very thin,gray] (\xMinMONTH,\i) -- (\xMaxMONTH,\i) node [right] (\xMinMONTH,\i) {};
    }

\draw [thick, black] (0,3) -- (\xMaxMONTH, 3);

\end{tikzpicture}


% ---------------------------------------------------------
% Draw the months
% --------------------------------------------------------- 

\newcommand*{\xMinLIFE}{0}%
\newcommand*{\xMaxLIFE}{18}%
\newcommand*{\yMinLIFE}{0}%
\newcommand*{\yMaxLIFE}{6}%


\Qitem{ \Qq{Draw a curve of how you experienced time in your life until now \\ up is fast, down is slow} }

\begin{tikzpicture}[thick,scale=0.5, every node/.style={scale=0.5}]

    \foreach \i in {\xMinLIFE,...,\xMaxLIFE} {
        \draw [very thin,gray] (\i,\yMinLIFE) -- (\i,\yMaxLIFE) ;
    }
    
     \foreach \x [count=\i] in {0, 5, 10, 15, 20, 25, 30, 35, 40, 45, 50, 55, 60, 65, 70, 75, 80, 85, 90} {
        \draw [very thin,gray] (\i-1, \yMinLIFE) -- (\i-1, \yMaxLIFE) node [below] at (\i-1, \yMinLIFE) {\x};
    }
    
    \foreach \i in {\yMinMONTH,...,\yMaxMONTH} {
        \draw [very thin,gray] (\xMinLIFE,\i) -- (\xMaxLIFE,\i) node [right] (\xMinLIFE,\i) {};
    }

\draw [thick, black] (0,3) -- (\xMaxLIFE, 3);

\end{tikzpicture}


% \foreach \x [count=\xi] in {a,...,e}

\end{document}